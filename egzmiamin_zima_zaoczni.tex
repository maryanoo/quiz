\documentclass[addpoints,11pt,a4paper]{exam}
\usepackage[utf8]{inputenc}
\usepackage[polish]{babel}


\title{Egzamin z Metod i technik badań społecznych}
\date{\today}
\author{dr Mariusz Piotrowski}

\begin{document}
	
	\maketitle
	\textbf{Imię i nazwisko:} \hrulefill \\\
	\textbf {Instrukcja}W pytaniach dopuszcza się możliwość wyboru więcej niż jednej odpowiedzi. Za każde pytanie można zdobyć maksymalnie 3 punkty. Aby uzyskać pełną punktację, należy zaznaczyć wszystkie prawidłowe odpowiedzi. Częściowo prawidłowe odpowiedzi są wliczane do punktacji proporcjonalnie. Zaznaczenie błędnej odpowiedzi powoduje nie naliczenie punktów za dane pytanie.
%	\textbf{Grupa:} \hrulefill
	
	\begin{questions}
		\section*{Część I: }
		
		\question Rozumowanie a priori w naukach społecznych jest:
		\begin{choices}
			\choice Nieprzydatne, bo są to nauki empiryczne
			\choice Przydatne, tylko na poziomie teorii
			\correctchoice Przydatne, na poziomie konstruowania hipotez 
			\choice Przydatne, gdyż wnioskowanie o zdarzeniach bierze się z ogólnych schematów poznawczych 
		\end{choices}
		
		
		
	\end{questions}
	
\section*{Punktacja i ocena}



\begin{center}
	\begin{tabular}{|c|c|}
		\hline
		Procent punktów & Ocena \\
		\hline
		< 50\% & ndst \\
		\hline
		>= 50\% & dst \\
		\hline
		>= 60\% & dst+ \\
		\hline
		>= 70\% & db \\
		\hline
		>= 80\% & db+ \\
		\hline
		>= 90\% & bdb \\
		\hline
	\end{tabular}
\end{center}



% Miejsce na ocenę z części II (otwarte) -  możesz dodać osobną tabelę lub po prostu wpisać ocenę
	
	
	
\end{document}
