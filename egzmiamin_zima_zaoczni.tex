\documentclass[addpoints,11pt,a4paper]{exam}
\usepackage[utf8]{inputenc}
\usepackage[polish]{babel}


\title{Egzamin z Metod i technik badań społecznych}
\date{\today}
\author{dr Mariusz Piotrowski}

\begin{document}
	
	\maketitle
	\textbf{Imię i nazwisko:} \hrulefill \\\
	\textbf {Instrukcja}W pytaniach dopuszcza się możliwość wyboru więcej niż jednej odpowiedzi. Za każde pytanie można zdobyć maksymalnie 3 punkty. Aby uzyskać pełną punktację, należy zaznaczyć wszystkie prawidłowe odpowiedzi. Częściowo prawidłowe odpowiedzi są wliczane do punktacji proporcjonalnie. Zaznaczenie błędnej odpowiedzi powoduje nie naliczenie punktów za dane pytanie.
%	\textbf{Grupa:} \hrulefill
	
	\begin{questions}
		\section*{Część I: }
		
		\question Rozumowanie a priori w naukach społecznych jest:
		\begin{choices}
			\choice Nieprzydatne, bo są to nauki empiryczne
			\choice Przydatne, tylko na poziomie teorii
			\correctchoice Przydatne, na poziomie konstruowania hipotez 
			\choice Przydatne, gdyż wnioskowanie o zdarzeniach bierze się z ogólnych schematów poznawczych 
		\end{choices}
		
			\question <<Nauka oferuje schematy pewne>>   to twierdzenie:
		\begin{choices}
			\correctchoice Prawdziwe, gdyż mogą istnieć twierdzenia pewne
			\choice Prawdziwe, gdy są oparte na twierdzeniach spostrzeżeniowych 
			\choice Nieprawdziwe, gdyż rozumowanie ludzkie nie opiera się na schematach pewnych
			\choice Nieprawdziwe, gdyż każde twierdzenie naukowe można obalić
		\end{choices}
		
			\question Warunkiem wystarczającym dla nauki jest:
		\begin{choices}
			\choice Intersubiektywna sprawdzalność
			\choice Intersubiektywna komunikowalność 
			\choice Zgodność z regułami logiki
			\correctchoice Zasada, że przekonanie z jakim głosimy jakieś twierdzenie nie może być większe niż jego uzasadnienie 
		\end{choices}
		
		\question Sprawdzanie twierdzeń w schemacie nomologiczno-dedukcyjnym jest:
		\begin{choices}
		\correctchoice Oparte na dedukcji z ogólnych zasad, testowanych empirycznie
		\choice Oparte na indukcji z danych empirycznych 
		\choice Oparte na logicznym rozumowaniu, bez potrzeby testów empirycznych \choice Oparte na teorii falsyfikacji i eksperymentalnym obalaniu hipotez
		\end{choices}
		 
		 	\question Teoria w naukach społecznych to:
		 \begin{choices}
		 	\choice Zbiór pojęć opisujących zjawisko
		 	\choice Zbiór pojęć wyjaśniających zjawisko 
		 	\choice Zbiór terminów opisujących zjawisko
		 	\correctchoice Zbiór twierdzeń wyjaśniających zjawisko przy użyciu terminów
		 \end{choices}
		 
		 	\question Teorie w naukach społecznych różnią się ze względu na :
		 \begin{choices}
		 	\choice Pojęcia opisujące to samo zjawisko
		 	\correctchoice Sposoby wyjaśnień tych samych zjawisk 
		 	\choice Obszar zainteresowań (życie codzienne, makrostruktury)
		 	\choice Zakres wyjaśnień (życie codzienne, makrostruktury)
		 \end{choices}
		 
		 \question Chłop polski w Ameryce i Polsce (Zaniecki i Thomas)  jest przykładem dzieła opartego o dane:
		  \begin{choices}
		 	\choice Statystyczne zestawienia, dotyczące migracji
		 	\correctchoice Standaryzowane wywiady z migrantami
		 	\choice Analizę treści dokumentów urzędowych
		 	\choice Uzyskane z dokumentów osobistych
		 \end{choices}
		 
		 \question Wskaźnik to zmienna, która jest:
		 \begin{choices} 
		 	\correctchoice Niedoskonałym reprezentantem pewnego pojęcia 
		 	\choice Zmienną, która opisuje tylko jedną cechę pojęcia 
		 	\choice Zmienną, która jest zawsze dokładnym odpowiednikiem pojęcia 
		 	\choice Zmienną, która opisuje tylko jedną cechę terminu
		 	\end{choices}
		 
		\question Co to jest hipoteza badawcza?
		\begin{choices}
			\choice Twierdzenie nie wymagające weryfikacji
			\correctchoice Twierdzenie wymagające weryfikacji
			\correctchoice Opis  relacji między badanymi zmiennymi
			\choice Pytanie wymagające weryfikacji
		\end{choices}
		
		\question Co to jest definicja operacyjna?
		\begin{choices}
			\choice Definicja zawierająca konotacje i denotacje pojęcia
			\correctchoice Definicja zawierająca opis operacji sprawdzających 
			\choice  Wyrażenie złożone z definiensa, definiendum i łącznika definicyjnego		
			\choice Inna nazwa rzetelnej definicji
		\end{choices}
		
		\question Zmienna zależna to:
		\begin{choices}
			\choice Zmienna manipulowana przez badacza
			\correctchoice Zmienna, na którą oddziaływuje zmienna niezależna
			\choice Każda zmienna w badaniu
			\choice Wynik końcowy badania
		\end{choices}
		
		\question Przykładem zmiennej nominalnej jest:
		\begin{choices}
			\choice Wiek
			\choice Dochód
			\correctchoice Płeć
			\choice Wybranie 5 na skali 5-stopniowej oceny atrakcyjności
		\end{choices}
		
			\question Przykładem zmiennej porządkowej jest:
		\begin{choices}
			\choice Wiek
			\choice Dochód
			\choice Płeć
			\correctchoice Wybranie 5 na skali 5-stopniowej oceny atrakcyjności
		\end{choices}
		
		\question Co charakteryzuje proces klasyfikacji w badaniach społecznych?
		\begin{choices} 
			\choice Proces polegający na przypisaniu jednostek do określonych grup na podstawie wspólnych cech 
			\correctchoice Proces polegający na przypisaniu jednostek do grup na podstawie wcześniej ustalonych kryteriów 
			\choice Proces organizowania danych według przypadkowych cech 
			\choice Proces zbierania danych dotyczących jednostek
			 \end{choices}
				
		\question Jakie jest główne założenie typologizacji w badaniach społecznych?
		\begin{choices} 
			\choice Typologizacja to przypisanie jednostek do jednolitych kategorii na podstawie ich indywidualnych cech 
			\correctchoice Typologizacja polega na tworzeniu systemu kategorii, które grupują jednostki w oparciu o złożone, wieloaspektowe cechy 
			\choice Typologizacja to proces przypisywania jednostek do jednej kategorii w oparciu o kryteria statystyczne 
			\choice Typologizacja to przypisanie jednostek do kategorii w oparciu o ich przynależność społeczną
			 \end{choices}
		
		\question Czym charakteryzują się skale szacunkowe w badaniach społecznych?
		\begin{choices} 
		\choice Są to narzędzia, które mierzą tylko jedną zmienną na poziomie nominalnym \correctchoice Są to narzędzia, które umożliwiają przypisanie wartości numerycznych do ocenianych cech lub zjawisk 
		\choice Są to narzędzia służące do klasyfikacji osób na podstawie ich cech społecznych \choice Są to narzędzia wykorzystywane tylko w badaniach eksperymentalnych \end{choices}
		
		\question Skala Likerta jest narzędziem badawczym, które służy do:
		\begin{choices} 
			\choice Mierzenia zmiennych nominalnych poprzez przypisanie kategorii 
			\choice Oceny częstości występowania danego zjawiska w populacji
			 \correctchoice Mierzenia postaw, opinii lub przekonań za pomocą sumatywnych ocen 
			 \choice Mierzenia zmiennych niezależnych w eksperymencie 
			 \end{choices}
			
			\question Skala dystansu społecznego Emory’ego Bogardusa służy do: 
			\begin{choices} 
				\choice Mierzenia stopnia zaawansowania technologicznego społeczeństw 
				\choice Oceny pozycji społecznej jednostek w hierarchii społecznej 
				\choice Mierzenia stopnia akceptacji lub dystansu wobec różnych grup społecznych 
				\correctchoice Oceny postaw i uprzedzeń wobec różnych grup społecznych  \end{choices} 
				
					\question Badanie reprezentatywne to badania, które: 
				\begin{choices}
					\choice Są przeprowadzane na wszystkich jednostkach populacji
					\correctchoice Wykorzystują próbę losową w celu uogólnienia wyników na całą populację
					\choice Obejmują badanie wyłącznie tych jednostek, które są typowe dla danej populacji
					\correctchoice Wykorzystują próbę, której cechy odzwierciedlają strukturę populacji
				\end{choices}
				
				\question Dobór kwotowy
				\begin{choices}
					\choice Metoda losowego doboru próby z populacji
					\choice Metoda doboru, w której próba jest odzwierciedleniem struktury populacji według określonych cech
					\choice Proces dobierania próby na podstawie kryteriów wygody i dostępności
					\correctchoice Proces doboru próby, w którym badacz ustala proporcje dla poszczególnych kategorii cech populacji
				\end{choices}
					\question W badaniach reprezentatywnych określono, że błąd standardowy wynosi 3,1\%. Jeśli partia A zdobyła 50\% poparcie, a partia B 5\% poparcie to wówczas:
				\begin{choices}
					\choice Błąd szacunku dla partii A i B jest taki sam i wynosi 3,1\%
					\correctchoice Błąd szacunku dla partii A wynosi 3.1\%, a dla partii B jest mniejszy.
					\choice Błąd szacunku dla partii A wynosi 3.1\%, a dla partii B jest większy.
					\choice Błąd szacunku dla partii B wynosi 3.1\%, a dla partii A jest mniejszy.
				\end{choices}
		
		\question Główną zaletą ankiety jest:
		\begin{choices}
			\choice Możliwość pogłębionej analizy
			\correctchoice Możliwość dotarcia do dużej liczby respondentów
			\choice Bezpośredni kontakt z badanym
			\choice Brak błędów pomiarowych
		\end{choices}
		
		\question W eksperymencie zmienna niezależna to:
		\begin{choices}
			\correctchoice Zmienna manipulowana przez badacza
			\choice Zmienna wynikowa
			\choice Zmienna kontrolna
			\choice Losowo wybrana zmienna
		\end{choices}
		
		\question Pytanie opisowe w badaniach społecznych ma na celu:
		\begin{choices}
			\choice Weryfikację hipotez
			\choice Odkrycie relacji przyczynowych
			\correctchoice Zrozumienie, jak wygląda dane zjawisko
			\choice Identyfikację zmiennych zależnych
		\end{choices}
		
		\question Trafność pomiaru oznacza:
		\begin{choices}
			\choice Powtarzalność wyników
			\correctchoice Zgodność pomiaru z rzeczywistością
			\choice Liczbę badanych zmiennych
			\choice Błąd losowy
		\end{choices}
		
		\question Operacjonalizacja to proces:
		\begin{choices}
			\choice Interpretacji wyników badania
			\correctchoice Przekształcania pojęć teoretycznych na zmienne i wskaźniki
			\choice Tworzenia hipotez
			\choice Zbierania danych
		\end{choices}
		
		\question Badanie eksploracyjne służy głównie:
		\begin{choices}
			\correctchoice Poszukiwaniu nowych zjawisk i hipotez
			\choice Potwierdzaniu znanych teorii
			\choice Opisowi danych
			\choice Weryfikacji zmiennych
		\end{choices}
		
		\question Główna różnica między badaniami ilościowymi a jakościowymi to:
		\begin{choices}
			\choice Sposób zbierania danych
			\correctchoice Rodzaj danych i sposób ich analizy
			\choice Liczba uczestników
			\choice Użycie technik statystycznych
		\end{choices}
		
		\section*{Część II: Pytania otwarte (maks. 10 punktów każde)}
		
		\question Sformułuj dwie przykładowe hipotezy badawcze na temat wpływu mediów społecznościowych na relacje międzyludzkie.
		\begin{solutionorlines}[5cm] % Miejsce na odpowiedź
		\end{solutionorlines}
		
		\question Omów różnice między wywiadem standaryzowanym a niestandaryzowanym. Podaj przykłady zastosowań obu technik.
		\begin{solutionorlines}[5cm] % Miejsce na odpowiedź
		\end{solutionorlines}
		
	\end{questions}
	
\section*{Punktacja i ocena}



\begin{center}
	\begin{tabular}{|c|c|}
		\hline
		Procent punktów & Ocena \\
		\hline
		< 50\% & ndst \\
		\hline
		>= 50\% & dst \\
		\hline
		>= 60\% & dst+ \\
		\hline
		>= 70\% & db \\
		\hline
		>= 80\% & db+ \\
		\hline
		>= 90\% & bdb \\
		\hline
	\end{tabular}
\end{center}



% Miejsce na ocenę z części II (otwarte) -  możesz dodać osobną tabelę lub po prostu wpisać ocenę
	
	
	
\end{document}